\chapter{Introducción}
La consola NES\footnote{\url{http://en.wikipedia.org/wiki/Nintendo_Entertainment_System}}fue una de las consola más vendidas de todos los tiempos. En 2009, la misma fue nombrada la mejor consola de videojuegos de la historia por IGN\footnote{\url{http://uk.ign.com/top-25-consoles/1.html}}.
Gracias a su popularidad y simplicidad muchas personas se interesaron en el funcionamiento de la consola, dando lugar a una abundante documentación al respecto.

Fue por estas dos razones que se dicidió emular está consola en particular.

Por otro lado, se eligió realizar un kernel para no depender de ningún sistema operativo. Esto introduce la posibilidad de ejecutar el emulador en máquinas más restringidas, que obviamente resultarían de menor costo que una computadora o la consola original.

Debido a la gran extensión del trabajo, se decidió seguir la filosofía de 'no reinventar la rueda'. El proyecto se construyó tomando como base otros kernels y emuladoras ya existentes, mencionados en la bibliografía (Capítulo ~\ref{chap:biblio}).

\section{Estructuración}

En un principio dividimos en dos partes principales el projecto.

Una es el \textbf{emulador} propiamente dicho, es decir la que se encarga de hacer todo lo que hacía internamente la consola.

La otra parte es el \textbf{kernel} encargada de inicializar todo lo necesario para que el emulador funcione, así como proveerle funciones de bajo nivel tales como escribir en pantalla o reservar memoria. Al no tener un sistema operativo detrás, funciones como malloc y free que cualquier programador de C supone siempre presentes deben ser implementadas por el kernel.
